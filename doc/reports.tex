%% Lab Report for EEET2493_labreport_template.tex
%% V1.0
%% 2019/01/16
%% This is the template for a Lab report following an IEEE paper. Modified by Francisco Tovar after Michael Sheel original document.


%% This is a skeleton file demonstrating the use of IEEEtran.cls
%% (requires IEEEtran.cls version 1.8b or later) with an IEEE
%% journal paper.
%%
%% Support sites:
%% http://www.michaelshell.org/tex/ieeetran/
%% http://www.ctan.org/pkg/ieeetran
%% and
%% http://www.ieee.org/

%%*************************************************************************
%% Legal Notice:
%% This code is offered as-is without any warranty either expressed or
%% implied; without even the implied warranty of MERCHANTABILITY or
%% FITNESS FOR A PARTICULAR PURPOSE! 
%% User assumes all risk.
%% In no event shall the IEEE or any contributor to this code be liable for
%% any damages or losses, including, but not limited to, incidental,
%% consequential, or any other damages, resulting from the use or misuse
%% of any information contained here.
%%
%% All comments are the opinions of their respective authors and are not
%% necessarily endorsed by the IEEE.
%%
%% This work is distributed under the LaTeX Project Public License (LPPL)
%% ( http://www.latex-project.org/ ) version 1.3, and may be freely used,
%% distributed and modified. A copy of the LPPL, version 1.3, is included
%% in the base LaTeX documentation of all distributions of LaTeX released
%% 2003/12/01 or later.
%% Retain all contribution notices and credits.
%% ** Modified files should be clearly indicated as such, including  **
%% ** renaming them and changing author support contact information. **
%%*************************************************************************

\documentclass[journal]{IEEEtran}

% *** CITATION PACKAGES ***
\usepackage[style=ieee,backend=bibtex]{biblatex} 
\bibliography{metrics_bib.bib}    %your file created using JabRef

% *** MATH PACKAGES ***
\usepackage{amsmath}

% *** PDF, URL AND HYPERLINK PACKAGES ***
\usepackage{url}
% correct bad hyphenation here
\hyphenation{op-tical net-works semi-conduc-tor}
\usepackage{graphicx}  %needed to include png, eps figures
\usepackage{float}  % used to fix location of images i.e.\begin{figure}[H]
\usepackage{enumitem}

\begin{document}

% paper title
\title{Report for "Wiskunde voor AI" course}

% author names 
\author{
Evert Jan Karman, nr invullen, %todo: Studentnummer invullen
John Stegink,  835 211 823
}% <-this % stops a space
        
% The report headers
\markboth{Report for "Wiskunde voor AI" course}%do not delete next lines
{Shell \MakeLowercase{\textit{et al.}}: Bare Demo of IEEEtran.cls for IEEE Journals}

% make the title area
\maketitle

% As a general rule, do not put math, special symbols or citations
% in the abstract or keywords.
\begin{abstract}
\end{abstract}

\begin{IEEEkeywords}
Software Quality Management, Metrics, Report
\end{IEEEkeywords}

\section{Introduction}
% Here we have the typical use of a "W" for an initial drop letter
% and "RITE" in caps to complete the first word.
% You must have at least 2 lines in the paragraph with the drop letter
% (should never be an issue)

\IEEEPARstart{T}{his} document describes the design choices and results for the Software Quality Management Lab assignment, which analyses the maintainability of two software projects SmallSQL\footnote{https://sourceforge.net/projects/smallsql/files/} and HyperSQL\footnote{https://sourceforge.net/projects/hsqldb/files/}. The assignment is divided into two parts. Firstly software metrics, volume, unit size, unit complexity and code duplication, are calculated for both projects. Secondly the calculated metrics are visualized so they can be interpreted easily.\\

The SIG maintainability model, as developed by I. Heitlager, T. Kuipers, and J. Visser~\cite{SIG}, was used for calculating the software metrics. RASCAL\footnote{https://www.rascal-mpl.org} was used to develop the scripts, this is a tool which supports all features needed for calculation and visualization of the metrics. The results were verified using SonarQube\footnote{https://www.sonarqube.org}, this is a widely used tool for measuring software metrics. The Community Edition is freely available, supports Java (the language in which both SmallSQL and HyperSQL were written), and calculates the metrics we needed. \\

The first part of the document describes the choices made for the software metrics. The results of visualization are described in the second part, together with the considerations.
\section{Software Metrics}

The first part of the assignment was about creating a program that can determine important metrics of a project. The metrics that were calculated were:
\begin{enumerate}[label=\Alph*.]
\item Lines of code
\item Unit size
\item Unit complexity
\item Code duplication
\end{enumerate}
Each metric is explained in a separate sub section.

\subsection{Lines of code}
The lines of code metric calculates the total number of lines per project. We chose to ignore all empty lines and all comments, because both of them are not real code. We discussed whether we would ignore import statements or lines with just one opening curly brace. We decided to count them because otherwise it would be more like counting statements than counting lines. Another argument was that I. Heitlager, T. Kuipers, and J. Visser~\cite{SIG} and SonarQube decided to ignore the empty lines and comments.\\

The empty lines and comments were removed by making use of regular expressions. Ignoring the comments is not quite as simple as it seems because it is possible that strings contain comments like \verb|/*| or \verb|//|. To solve this problem we first replaced all strings with a dummy string, then we removed the comments and replaced the dummy string with the original string. This 'cleaning' of code is used for the unit size and code duplication metrics too.\\

When the results were compared with the results in SonarQube, the number of lines in SmallSQL were equal. This is not surprising because we used the same criteria as SonarCube. However the number of lines counted for HyperSQL are smaller than the number of lines in our program. We found out that the reason for this is that SonarQube does not handle nested comments correctly.



\subsection{Unit size}
..
\subsection{Unit complexity}
Unity complexity was measured with the method described by T.J. McCabe~\cite{complexity} by measuring the Cyclomatic Complexity. Because the statements of the code had to be analyzed instead of the code text (as with the other measures) we didn't use regular expression but an abstract syntax tree. We counted all statements that could branch the code (including the \verb|? :|, \verb|&&| and \verb"||" expressions).\\

We chose to adhere to the method described by T.J. McCabe~\cite{complexity} though SonarQube slightly deviates from this method\footnote{https://docs.sonarqube.org/7.3/Metrics-Complexity.html}. SonarQube does not count \verb|else| and \verb|default| for example, but it does count a \verb|return| if is not at the end of a method. These differences made it harder for us to compare our results with the results of SonarQube.




%you can use a table generator from here: https://www.tablesgenerator.com/#



\appendices
\section{Results of the metrics}
\subsection{SmallSQL}
\begin{verbatim}
    Small SQL
    ----
    Lines of code: 19346
    Unit size: 
      Simple 87%
      Moderate 7%
      High 2%
      Very high 1%
    Complexity: 
      Simple 93%
      Moderate 3%
      High 2%
      Very high 0%
    Duplication: 8%
    
    volume score:  +
    unit size score:  -
    unit complexity score:  +
    duplicaton score:  +/-
    
    analysability score:  +/-
    changability score:  +
    testability score:  +/-

    overal maintainability score:  +/-
\end{verbatim}

\subsection{HyperSQL}
\begin{verbatim}
    HyperSQL
    ----
    Lines of code: 146691
    Unit size: 
      Simple 76%
      Moderate 11%
      High 6%
      Very high 3%
    Complexity: 
      Simple 89%
      Moderate 5%
      High 2%
      Very high 0%
    Duplication: 19%

    volume score:  -
    unit size score:  -
    unit complexity score:  +
    duplicaton score:  -

    analysability score:  -
    changability score:  +/-
    testability score:  +/-

    overal maintainability score:  +/-
\end{verbatim}



% references section

% can use a bibliography generated by BibTeX as a .bbl file
% BibTeX documentation can be easily obtained at:
% http://mirror.ctan.org/biblio/bibtex/contrib/doc/
% The IEEEtran BibTeX style support page is at:
% http://www.michaelshell.org/tex/ieeetran/bibtex/
%\bibliographystyle{IEEEtran}
% argument is your BibTeX string definitions and bibliography database(s)
%\bibliography{IEEEabrv,../bib/paper}
%
% <OR> manually copy in the resultant .bbl file
% set second argument of \begin to the number of references
% (used to reserve space for the reference number labels box)

%use following command to generate the list of cited references

\printbibliography


% that's all folks
\end{document}


